\documentclass[a4paper, 12pt]{book}
%\documentclass[a4paper, 12pt, draft]{book}  Nalogo preverite tudi z opcijo draft, ki vam bo pokazala, katere vrstice so predolge!



\usepackage[utf8x]{inputenc}   % omogoča uporabo slovenskih črk kodiranih v formatu UTF-8
\usepackage[slovene,english]{babel}    % naloži, med drugim, slovenske delilne vzorce
\usepackage[pdftex]{graphicx}  % omogoča vlaganje slik različnih formatov
\usepackage{fancyhdr}          % poskrbi, na primer, za glave strani
\usepackage{amssymb}           % dodatni simboli
\usepackage{amsmath}           % eqref, npr.
%\usepackage{hyperxmp}
\usepackage[hyphens]{url}  % dodal Solina
\usepackage{comment}       % dodal Solina

\usepackage[pdftex, colorlinks=true,
						citecolor=black, filecolor=black, 
						linkcolor=black, urlcolor=black,
						pagebackref=false, 
						pdfproducer={LaTeX}, pdfcreator={LaTeX}, hidelinks]{hyperref}

\usepackage{color}       % dodal Solina
\usepackage{soul}       % dodal Solina

%%%%%%%%%%%%%%%%%%%%%%%%%%%%%%%%%%%%%%%%
%	DIPLOMA INFO
%%%%%%%%%%%%%%%%%%%%%%%%%%%%%%%%%%%%%%%%
\newcommand{\ttitle}{Zasnova in razvoj platforme za spremljanje metrik HTML5 spletnih aplikacij}
\newcommand{\ttitleEn}{Design and implementation of platform for monitoring metrics of HTML5 web applications}
\newcommand{\tsubject}{\ttitle}
\newcommand{\tsubjectEn}{\ttitleEn}
\newcommand{\tauthor}{Miha Jamšek}
\newcommand{\tkeywords}{metrike, html, html5, splet, spletne aplikacije, spa, platforma}
\newcommand{\tkeywordsEn}{metrics, html, html5, web, web applications, spa, platform}


%%%%%%%%%%%%%%%%%%%%%%%%%%%%%%%%%%%%%%%%
%	HYPERREF SETUP
%%%%%%%%%%%%%%%%%%%%%%%%%%%%%%%%%%%%%%%%
\hypersetup{pdftitle={\ttitle}}
\hypersetup{pdfsubject=\ttitleEn}
\hypersetup{pdfauthor={\tauthor, mj6243@student.uni-lj.si}}
\hypersetup{pdfkeywords=\tkeywordsEn}


 


%%%%%%%%%%%%%%%%%%%%%%%%%%%%%%%%%%%%%%%%
% postavitev strani
%%%%%%%%%%%%%%%%%%%%%%%%%%%%%%%%%%%%%%%%  

\addtolength{\marginparwidth}{-20pt} % robovi za tisk
\addtolength{\oddsidemargin}{40pt}
\addtolength{\evensidemargin}{-40pt}

\renewcommand{\baselinestretch}{1.3} % ustrezen razmik med vrsticami
\setlength{\headheight}{15pt}        % potreben prostor na vrhu
\renewcommand{\chaptermark}[1]%
{\markboth{\MakeUppercase{\thechapter.\ #1}}{}} \renewcommand{\sectionmark}[1]%
{\markright{\MakeUppercase{\thesection.\ #1}}} \renewcommand{\headrulewidth}{0.5pt} \renewcommand{\footrulewidth}{0pt}
\fancyhf{}
\fancyhead[LE,RO]{\sl \thepage} 
%\fancyhead[LO]{\sl \rightmark} \fancyhead[RE]{\sl \leftmark}
\fancyhead[RE]{\sc \tauthor}              % dodal Solina
\fancyhead[LO]{\sc Diplomska naloga}     % dodal Solina


\newcommand{\BibTeX}{{\sc Bib}\TeX}

%%%%%%%%%%%%%%%%%%%%%%%%%%%%%%%%%%%%%%%%
% naslovi
%%%%%%%%%%%%%%%%%%%%%%%%%%%%%%%%%%%%%%%%  


\newcommand{\autfont}{\Large}
\newcommand{\titfont}{\LARGE\bf}
\newcommand{\clearemptydoublepage}{\newpage{\pagestyle{empty}\cleardoublepage}}
\setcounter{tocdepth}{1}	      % globina kazala

%%%%%%%%%%%%%%%%%%%%%%%%%%%%%%%%%%%%%%%%
% konstrukti
%%%%%%%%%%%%%%%%%%%%%%%%%%%%%%%%%%%%%%%%  
\newtheorem{izrek}{Izrek}[chapter]
\newtheorem{trditev}{Trditev}[izrek]
\newenvironment{dokaz}{\emph{Dokaz.}\ }{\hspace{\fill}{$\Box$}}

%%%%%%%%%%%%%%%%%%%%%%%%%%%%%%%%%%%%%%%%%%%%%%%%%%%%%%%%%%%%%%%%%%%%%%%%%%%%%%%
%% PDF-A
%%%%%%%%%%%%%%%%%%%%%%%%%%%%%%%%%%%%%%%%%%%%%%%%%%%%%%%%%%%%%%%%%%%%%%%%%%%%%%%


%%%%%%%%%%%%%%%%%%%%%%%%%%%%%%%%%%%%%%%% 
% define medatata
%%%%%%%%%%%%%%%%%%%%%%%%%%%%%%%%%%%%%%%% 
\def\Title{\ttitle}
\def\Author{\tauthor, matjaz.kralj@fri.uni-lj.si}
\def\Subject{\ttitleEn}
\def\Keywords{\tkeywordsEn}

%%%%%%%%%%%%%%%%%%%%%%%%%%%%%%%%%%%%%%%% 
% \convertDate converts D:20080419103507+02'00' to 2008-04-19T10:35:07+02:00
%%%%%%%%%%%%%%%%%%%%%%%%%%%%%%%%%%%%%%%% 
\def\convertDate{%
    \getYear
}

{\catcode`\D=12
 \gdef\getYear D:#1#2#3#4{\edef\xYear{#1#2#3#4}\getMonth}
}
\def\getMonth#1#2{\edef\xMonth{#1#2}\getDay}
\def\getDay#1#2{\edef\xDay{#1#2}\getHour}
\def\getHour#1#2{\edef\xHour{#1#2}\getMin}
\def\getMin#1#2{\edef\xMin{#1#2}\getSec}
\def\getSec#1#2{\edef\xSec{#1#2}\getTZh}
\def\getTZh +#1#2{\edef\xTZh{#1#2}\getTZm}
\def\getTZm '#1#2'{%
    \edef\xTZm{#1#2}%
    \edef\convDate{\xYear-\xMonth-\xDay T\xHour:\xMin:\xSec+\xTZh:\xTZm}%
}

\expandafter\convertDate\pdfcreationdate 

%%%%%%%%%%%%%%%%%%%%%%%%%%%%%%%%%%%%%%%%
% get pdftex version string
%%%%%%%%%%%%%%%%%%%%%%%%%%%%%%%%%%%%%%%% 
\newcount\countA
\countA=\pdftexversion
\advance \countA by -100
\def\pdftexVersionStr{pdfTeX-1.\the\countA.\pdftexrevision}


%%%%%%%%%%%%%%%%%%%%%%%%%%%%%%%%%%%%%%%%
% XMP data
%%%%%%%%%%%%%%%%%%%%%%%%%%%%%%%%%%%%%%%%  
\usepackage{xmpincl}
\includexmp{pdfa-1b}

%%%%%%%%%%%%%%%%%%%%%%%%%%%%%%%%%%%%%%%%
% pdfInfo
%%%%%%%%%%%%%%%%%%%%%%%%%%%%%%%%%%%%%%%%  
\pdfinfo{%
    /Title    (\ttitle)
    /Author   (\tauthor, mj6243@student.uni-lj.si)
    /Subject  (\ttitleEn)
    /Keywords (\tkeywordsEn)
    /ModDate  (\pdfcreationdate)
    /Trapped  /False
}


%%%%%%%%%%%%%%%%%%%%%%%%%%%%%%%%%%%%%%%%%%%%%%%%%%%%%%%%%%%%%%%%%%%%%%%%%%%%%%%
%%%%%%%%%%%%%%%%%%%%%%%%%%%%%%%%%%%%%%%%%%%%%%%%%%%%%%%%%%%%%%%%%%%%%%%%%%%%%%%

\begin{document}
\selectlanguage{slovene}
\frontmatter
\setcounter{page}{1} %
\renewcommand{\thepage}{}       % preprecimo težave s številkami strani v kazalu
\newcommand{\sn}[1]{"`#1"'}                    % dodal Solina (slovenski narekovaji)

%%%%%%%%%%%%%%%%%%%%%%%%%%%%%%%%%%%%%%%%
%naslovnica
 \thispagestyle{empty}%
   \begin{center}
    {\large\sc Univerza v Ljubljani\\%
      Fakulteta za računalništvo in informatiko}%
    \vskip 10em%
    {\autfont \tauthor\par}%
    {\titfont \ttitle \par}%
    {\vskip 3em \textsc{DIPLOMSKO DELO\\[5mm]         % dodal Solina za ostale študijske programe
%    VISOKOŠOLSKI STROKOVNI ŠTUDIJSKI PROGRAM\\ PRVE STOPNJE\\ RAČUNALNIŠTVO IN INFORMATIKA}\par}%
    UNIVERZITETNI  ŠTUDIJSKI PROGRAM\\ PRVE STOPNJE\\ RAČUNALNIŠTVO IN INFORMATIKA}\par}%
%    INTERDISCIPLINARNI UNIVERZITETNI\\ ŠTUDIJSKI PROGRAM PRVE STOPNJE\\ RAČUNALNIŠTVO IN MATEMATIKA}\par}%
%    INTERDISCIPLINARNI UNIVERZITETNI\\ ŠTUDIJSKI PROGRAM PRVE STOPNJE\\ UPRAVNA INFORMATIKA}\par}%
%    INTERDISCIPLINARNI UNIVERZITETNI\\ ŠTUDIJSKI PROGRAM PRVE STOPNJE\\ MULTIMEDIJA}\par}%
    \vfill\null%
    {\large \textsc{Mentor}: prof. dr. Matjaž B. Jurič\par}%
    {\vskip 2em \large Ljubljana, 2019 \par}%
\end{center}
% prazna stran
%\clearemptydoublepage      % dodal Solina (izjava o licencah itd. se izpiše na hrbtni strani naslovnice)

%%%%%%%%%%%%%%%%%%%%%%%%%%%%%%%%%%%%%%%%
%copyright stran
\thispagestyle{empty}
\vspace*{8cm}

\noindent
{\sc Copyright}. 
Rezultati diplomske naloge so intelektualna lastnina avtorja in Fakultete za računalništvo in informatiko Univerze v Ljubljani.
Za objavo in koriščenje rezultatov diplomske naloge je potrebno pisno privoljenje avtorja, Fakultete za računalništvo in informatiko ter mentorja.
Izvorna koda diplomskega dela, njeni rezultati in v ta namen razvita programska oprema je ponujena pod odprtokodno licenco MIT. Podrobnosti licence so dostopne na spletni strani https://opensource.org/licenses/MIT.

\begin{center}
\mbox{}\vfill
\emph{Besedilo je oblikovano z urejevalnikom besedil \LaTeX.}
\end{center}
% prazna stran
\clearemptydoublepage

%%%%%%%%%%%%%%%%%%%%%%%%%%%%%%%%%%%%%%%%
% stran 3 med uvodnimi listi
\thispagestyle{empty}
\vspace*{4cm}

\noindent
Fakulteta za računalništvo in informatiko izdaja naslednjo nalogo:
\medskip
\begin{tabbing}
\hspace{32mm}\= \hspace{6cm} \= \kill




Tematika naloge:
\end{tabbing}
TODO
\vspace{15mm}






\vspace{2cm}

% prazna stran
\clearemptydoublepage

% zahvala
\thispagestyle{empty}\mbox{}\vfill\null\it%
\noindent
TODO
\rm\normalfont

% prazna stran
\clearemptydoublepage

%%%%%%%%%%%%%%%%%%%%%%%%%%%%%%%%%%%%%%%%

% prazna stran
\clearemptydoublepage


%%%%%%%%%%%%%%%%%%%%%%%%%%%%%%%%%%%%%%%%
% kazalo
\pagestyle{empty}
\def\thepage{}% preprecimo tezave s stevilkami strani v kazalu
\tableofcontents{}


% prazna stran
\clearemptydoublepage

%%%%%%%%%%%%%%%%%%%%%%%%%%%%%%%%%%%%%%%%
% seznam kratic

\chapter*{Seznam uporabljenih kratic}  % spremenil Solina, da predolge vrstice ne gredo preko desnega roba

\begin{comment}

\begin{tabular}{l|l|l}
  {\bf kratica} & {\bf angleško} & {\bf slovensko} \\ \hline
  % after \\: \hline or \cline{col1-col2} \cline{col3-col4} ...
  {\bf CA} & classification accuracy & klasifikacijska točnost \\
  {\bf DBMS} & database management system & sistem za upravljanje podatkovnih baz \\
  {\bf SVM} & support vector machine & metoda podpornih vektorjev \\
  \dots & \dots & \dots \\
\end{tabular}
\end{comment}

\noindent\begin{tabular}{p{0.1\textwidth}|p{.4\textwidth}|p{.4\textwidth}}    % po potrebi razširi prvo kolono tabele na račun drugih dveh!
  {\bf kratica} & {\bf angleško}                             & {\bf slovensko} \\ \hline
  {\bf CA}      & classification accuracy               & klasifikacijska točnost \\
  {\bf DBMS} & database management system & sistem za upravljanje podatkovnih baz \\
  {\bf SVM}   & support vector machine              & metoda podpornih vektorjev \\
%  \dots & \dots & \dots \\
\end{tabular}


% prazna stran
\clearemptydoublepage

%%%%%%%%%%%%%%%%%%%%%%%%%%%%%%%%%%%%%%%%
% povzetek
\addcontentsline{toc}{chapter}{Povzetek}
\chapter*{Povzetek}

\noindent\textbf{Naslov:} \ttitle
\bigskip

\noindent\textbf{Avtor:} \tauthor
\bigskip

%\noindent\textbf{Povzetek:} 
\noindent TODO

\bigskip

\noindent\textbf{Ključne besede:} \tkeywords.
% prazna stran
\clearemptydoublepage

%%%%%%%%%%%%%%%%%%%%%%%%%%%%%%%%%%%%%%%%
% abstract
\selectlanguage{english}
\addcontentsline{toc}{chapter}{Abstract}
\chapter*{Abstract}

\noindent\textbf{Title:} \ttitleEn
\bigskip

\noindent\textbf{Author:} \tauthor
\bigskip

%\noindent\textbf{Abstract:} 
\noindent TODO
\bigskip

\noindent\textbf{Keywords:} \tkeywordsEn.
\selectlanguage{slovene}
% prazna stran
\clearemptydoublepage

%%%%%%%%%%%%%%%%%%%%%%%%%%%%%%%%%%%%%%%%
\mainmatter
\setcounter{page}{1}
\pagestyle{fancy}

\chapter{Uvod}

Dandanes je digitalna ekonomija že zelo razširjen pojav. Velik del podjetij vsaj delno posluje digitalno, za marsikatero pa to predstavlja tudi glavni poslovni model. Ker je eden od glavnih gradnikov digitalne ekonomije tudi e-poslovna infrastruktura \cite{digital_econ}, morajo podjetja nameniti temu tudi dovoljšen del pozornosti.

Predvsem pri poslovnih modelih, ki se vrtijo okoli spletne strani – to so razne spletne trgovine in ostale storitve, ki so na voljo skozi spletne vmesnike – je uporabniška izkušnja eden izmed najpomembnejših faktorjev, s katerim uporabnika pritegnemo k uporabi naše storitve in ga tam tudi zadržimo. Uporabnik, ki ima na voljo dve razmeroma podobni storitvi, bo izbral  tisto, ki ima preglednejši uporabniški vmesnik in se hitreje odziva na uporabnikove zahteve. 

Ker pa se uporabnikova pričakovanja in poslovne zahteve neprestano večajo, so z njimi pričele naraščati tudi velikosti spletnih strani. Razvijalci so tako spoznali omejitve trenutnih spletnih tehnologij, saj se je pojavljalo veliko redundantne kode, datoteke so postajale vedno večje, posledično težje za vzdrževanje in bilo je veliko prepletanja med strežniško in klientovo kodo.

Eden izmed odgovorov na to problematiko se je pojavil v obliki t.i. spletnih aplikacij na eni strani oz. single page application (nadalje SPA) v angleščini. To so aplikacije, kjer se uporabniški vmesnik zgradi šele, ko je stran naložena, torej najprej se prenesejo vsi gradniki (Javascript, HTML, CSS datoteke), nakar pa se aplikacija zažene in prikaže našo vsebino. Struktura teh aplikacij nam omogoča lažjo organizacijo kode, saj podpirajo konsolidiranje redundantne kode v ti. komponente, ki jih lahko nato uporabljamo skozi celotno aplikacijo. Te lahko v celoti implementiramo sami, najpogosteje pa se za to uporablja eno od ogrodij, od katerih so trenutno najpopularnejše Angular, ReactJS in pa VueJs.

Seveda pa njihov doprinos ni samo na strani razvijalcev, ampak tudi uporabnikov. Že samo ime, aplikacija na eni strani, nam pove, da taka aplikacija ne uporablja običajne navigacije.

Pri klasičnem pristopu, smo imeli za vsako stran svoj HTML dokument, ki se je prenesel ko smo zahtevali to stran. Nasprotno, imamo pri SPA samo en HTML dokument. Vsa nadaljnja navigacija je »mehka«, ker se ne prenese nov HTML dokument, ampak se nova stran izriše s pomočjo Javascripta.

Tako, sicer za ceno počasnejšega prvega nalaganja, pospešimo vse nadaljnje zahtevke, in damo uporabniku občutek, da je stran bolj odzivna. Nadalje, kjer smo prej podatke izpisali na stran s pomočjo strežniškega renderinga, sedaj te podatke pridobimo z AJAX zahtevki. Razlika za uporabnika je v tem, da lahko vidi že stran, čeprav vsebine še ni. Spet, tako damo občutek, da je spletna stran hitrejša.

Zaradi razlik v delovanju takih aplikacij od klasičnih, so metrike, ki smo jih spremljali pri klasičnih, neprimerne. Primer ene take metrike je ti. PLT (page load time) oziroma čas nalaganja strani. Ker moramo pri SPA ločiti »trdo« od »mehke« navigacije (saj »trda« vsebuje tudi zagonski čas aplikacije, ki je še ena od metrik, ki ni prisotna pri klasičnih straneh), moramo tukaj postopati drugače.

Zato se v tej diplomski nalogi ukvarjam z zasnovo platforme, ki spremlja metrike, primerne za SPA aplikacije.

Te metrike, razvijalcem omogočajo prepoznati »ozko grlo« njihovih aplikacij, da jih lahko odpravijo ter tako omogočijo svojim uporabnikom boljšo izkušnjo.



\chapter{Spremljanje uporabniških metrik v realnem času}
\label{ch0}
Kaj spremljanje metrik sploh pomeni? Spremljanje metrik je proces, pri katerem zbiramo razne podatke o naši aplikaciji (ta ne rabi biti spletna), kateri nam dajo povratno informacijo o kakovosti naše aplikacije. Te metrike lahko spremljajo kakovost uporabniškega vmesnika, odzivnost aplikacije,  kvaliteto povezave naših uporabnikov, ali pa še kaj drugega. Nekega splošnega recepta za seznam metrik, ki jih moramo zajemati, ni. Spremljamo pač tiste, ki so za nas najpomembnejše.

Tako imamo danes veliko načinov za spremljanje metrik, vsi se pa v grobem delijo na dve kategoriji: sintetično in pasivno spremljanje.

Sintetično spremljanje je spremljanje, kjer s pomočjo orodij simuliramo uporabnike in njihove akcije. S tem spremljanjem lahko zgodaj odkrijemo napake v funkcionalnosti naše aplikacije. Ravno tako odkrijemo kakšne neoptimizirane dele kode, ki upočasnjujejo delovanje aplikacije.

Ima pa sintetično spremljanje en velik problem in sicer, orodja s katerimi testiramo, ne morejo predvideti vseh možnih poti, ki jih uporabnik izbere po naši aplikaciji. Običajno se tako osredotočimo samo na tisti pravi potek dela, kot smo si ga zamislili ko smo načrtovali aplikacijo. Ker pa uporabniki ne poznajo pravega poteka dela, velikokrat izberejo drugačno pot do željene storitve, ki je s sintetičnim spremljanjem nismo zajeli. Ta problem rešujemo z uporabo druge kategorije – tj.  pasivno spremljanje.

Pasivno spremljanje je spremljanje dejanskega prometa, ki ga imamo v naši aplikaciji. Torej to je spremljanje, ki ga ne izvajamo več v varnem lokalnem razvojnem okolju, oziroma testnih strežnikih, ampak se izvaja v produkcijskem okolju. Ta vrsta spremljanja zajame vse probleme aplikacije, vendar jih zajame šele takrat ko so se le-ti že zgodili.

Ena od tehnologij pasivnega spremljanja je ti. spremljanje uporabniških metrik v realnem času oziroma RUM (Real User Monitoring) v angleščini. S temi uporabniškimi metrikami, lahko nato zaznamo neželene pojave kot so pogoste napake ali upočasnitve. Zaznamo lahko odzivnost našega uporabniškega vmesnika, kjer spremljamo preklop med stanji aplikacije, pa tudi njegovo intuitivnost, tako, da spremljamo premik miške po zaslonu in vidimo, ali se uporabnik »lovi« po zaslonu. Poleg tehničnih metrik, pa lahko spremljamo še ostale metrike, kot so število obiskov strani, ali se stranke vračajo na našo stran, kakšen čas v povprečju preživijo na njej, ipd.

Veliko teh metrik lahko spremljamo z uporabo orodij kot so Google Analytics \cite{ga_website}, predvsem prej omenjene ne-tehnične metrike. Google Analytics je sicer orodje pisano za uporabo v klasičnih več-stranskih spletnih straneh, kjer se vsaka stran posebej naloži in takrat sproži dogodek, ki ga GA zabeleži, vendar lahko njegovo uporabo priredimo tako, da deluje tudi v aplikacijah na eni strani.

V primeru da želimo spremljati tehnične metrike naše aplikacije na eni strani, pa hitro ugotovimo, da nam GA ne zadošča. Tukaj je iniciativo prevzel Microsoft, z razvojem orodja Mezzurite \cite{mezzurite_website}, (v času pisanja je bil v verziji 1) ki ponuja zbiranje metrik za ogrodja Angular, AngularJS in React.

Pri tem je Microsoft naletel na problem, in sicer ta, da so navigacija in življenjski cikel aplikacije zelo odvisni od ogrodja, ki ga uporabljamo. Njihova rešitev je bila ta, da so podprli samo tri ogrodja, in v zameno olajšali delo razvijalcu, ki uporablja njihov produkt. Dolgih nosov so tako ostali razvijalci, ki uporabljajo npr. Vue.js (še eno izmed popularnih ogrodij). Ker pa se popularnost ogrodij in tehnologij zelo hitro spreminja, je pri takem pristopu treba biti neprestano obveščen o spremembah, pa tudi te spremembe poznati v globino, saj je potrebno poznati ogrodje precej dobro, da vemo kam umestiti naše spremljanje metrik. Tega se verjetno zaveda tudi Microsoft, saj je naznanil 2. verzijo Mezzurite platforme, kjer bo najverjetneje spremenil pristop.

Sam sem v zasnovi moje implementacije storil obratno, kot je storil Microsoft in sicer, zasnoval sem generično rešitev, ki deluje neodvisno od uporabljenega ogrodja. Ta zahteva nekaj več dela s strani razvijalca in zahteva, da razvijalec sam pozna ogrodje, ki ga uporablja, ampak skrati mu omogoča, da prilagodi spremljanje metrik tako, da mu bo dalo karseda točne metrike, primerne za njegovo aplikacijo.

\chapter{Razlike pri spremljanju metrik spletne aplikacije na eni strani (SPA) in klasične spletne strani}
\label{ch1}

\begin{equation}
\label{eq-01}
PLT = ALT + VLT
\end{equation}

\chapter{Zajete metrike}
\label{ch2}

\section{Sledenje premikov miške uporabnika}

\section{Zagon aplikacije}

\section{Nalagalni čas posamezne strani}

\section{Nalaganje gradnikov strani}

\chapter{Zasnova generične rešitve za zbiranje in spremljanje uporabniških metrik}
\label{ch3}

\section{Načrt}

\section{Implementacija}

\section{Prototip}

\chapter{Sklepne ugotovitve}
\label{ch4}

\section{Nadaljnje delo}

\newpage %dodaj po potrebi, da bo številka strani za Literaturo v Kazalu pravilna!
\ \\
\clearpage
\addcontentsline{toc}{chapter}{Literatura}
\bibliographystyle{plain}
\bibliography{literatura}


\end{document}

